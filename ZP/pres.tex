\documentclass[14pt]{beamer}
\usepackage[utf8x]{inputenc}
\usepackage[T1]{fontenc}
\usepackage[ngerman]{babel}
\usetheme[secheader]{Boadilla}
\usefonttheme{serif}

\title{3D-Tumorvisualisation}
\subtitle{Zwischenpräsentation}
\author{Uli Köhler}
\institute[EMG]{Ernst-Mach-Gymnasium Haar}
\date{8.~Juli 2010}

\AtBeginSection[]{} % for optional outline or other recurrent slide

\begin{document}
\frame{\titlepage}
\frame{\tableofcontents}
\begin{frame}
   \frametitle{Inhalt}
   \begin{itemize}
    \item  Behandelte Hauptkonzepte
    \begin{itemize}
      \item Hervorhebung von Tumoren
      \item Generierung von Zwischenbildern
      \item Interaktive Navigation im 3D-Koordinatensystem
    \end{itemize}
    \pause
    \item Projekt: MediGL
    \begin{itemize}
     \item Architektur
     \item Fortschritt
    \end{itemize}
   \end{itemize}
\end{frame}
 \section{Behandelte Hauptkonzepte}
%%%%%%
\subsection{Hervorhebung von Tumoren}
\begin{frame}[allowframebreaks]
 \frametitle{Hervorhebung von Tumoren}
    \begin{itemize}
     \item \textbf{Idee:} Mediziner sollen Tumoren schnell erkennen können
     \item Automatisierte Erkennung anhand der HUs\\
	  $\Rightarrow$ Algorithmus: Abbildung der Hounsfield-Skala auf einen Farbverlauf
     \item Grenzen des Verfahrens:
      \begin{itemize}
	\item Tumoren mit geringem Kontrast zum umliegenden Gewebe
	\item Menschliche Kenntnis zur Zuordnung von anatomischen Strukturen nötig
      \end{itemize}
    \end{itemize}
    \pgfimage[width=\textwidth]{F36-rendered.png}
\end{frame}
%%%%%%
\subsection{Generierung von Zwischenbildern}

\begin{frame}[allowframebreaks]
 \frametitle{Generierung von Zwischenbildern}
    \begin{itemize}
     \item \textbf{Idee:} Mediziner sollen Tumoren schnell erkennen können
     \item Automatisierte Erkennung anhand der HUs\\
	  $\Rightarrow$ Algorithmus: Abbildung der Hounsfield-Skala auf einen Farbverlauf
     \item Grenzen des Verfahrens:
      \begin{itemize}
	\item Tumoren mit geringem Kontrast zum umliegenden Gewebe
	\item Menschliche Kenntnis zur Zuordnung von anatomischen Strukturen nötig
      \end{itemize}
    \end{itemize}
    \pgfimage[width=\textwidth]{interimage1.jpg}
\end{frame}
\end{document}
