\documentclass[14pt]{beamer}
\usepackage[utf8x]{inputenc}
\usepackage[T1]{fontenc}
\usepackage[ngerman]{babel}
\usepackage{hyperref}
\usetheme[secheader]{Boadilla}
\usefonttheme{serif}

\title{3D-Tumorvisualisation}
\subtitle{Endpräsentation}
\author{Uli Köhler}
\institute[EMG]{Ernst-Mach-Gymnasium Haar}
\date{24.~Januar 2011}


\AtBeginSection[]{} % for optional outline or other recurrent slide

\begin{document}
\frame{\titlepage}
\begin{frame}
\frametitle{Aufbau dieser Präsentation}
\tableofcontents
\end{frame}
\section{Kernkonzepte}
%%%%%%
\subsection{Hervorhebung von Tumoren}
\begin{frame}[allowframebreaks]
 \frametitle{Hervorhebung von Tumoren}
    \begin{itemize}
     \item \textbf{Idee:} Mediziner sollen Tumoren schnell erkennen können
     \item Automatisierte Erkennung anhand der HUs\\
	  $\Rightarrow$ Algorithmus: Abbildung der Hounsfield-Skala auf einen Farbverlauf (`Klassifizierung`)
     \item Grenzen des Verfahrens:
      \begin{itemize}
	\item Tumoren mit geringem Kontrast zum umliegenden Gewebe
	\item Menschliche Kenntnis zur Zuordnung von anatomischen Strukturen nötig
      \end{itemize}
    \end{itemize}
    \pgfimage[width=\textwidth]{F36-rendered.png}
\end{frame}
%%%%%%
\subsection{Plattformen zur Berechnung der Visualisationsdaten}
\begin{frame}
 \frametitle{Plattformen zur Berechnung der Visualisationsdaten}
\vspace{8mm}	
 \begin{center}
 \pgfimage[width=0.8\textwidth]{platforms-comp.png}
 \end{center}
\end{frame}

%%%%%%
%
%
%
\section{Projekt: VERTEBRA}
\begin{frame}
\frametitle{VERTEBRA: Funktionalität}
\begin{itemize}
 \item VERTEBRA = \textbf{V}olumetric \textbf{E}xaminer for \textbf{R}adiological/\textbf{T}omographical \textbf{E}xperimental \textbf{B}asic \textbf{R}ealtime \textbf{A}nalysis
 \item Ziel: Demonstration eines Klassifizierungsalgorithmus
 \item Berechnung auf der CPU / Rendering auf der GPU $\rightarrow$ Nur Standardhardware nötig
 \item DICOM-Import
 \item Hounsfield-Fenster
 \item Nur Demonstration - keine Produktivsoftware!
\end{itemize}
\end{frame}
%
%
\section{Augmented Reality}
\begin{frame}[allowframebreaks]
\frametitle{Augmented Reality}
\begin{columns}
        \column{.45\textwidth}
	    \begin{itemize}
	    \item `Erweiterte Realität`
	    \item Echtzeiteinblendung von Informationen, z.B. für Chirurgen
	    \item Beispiel: Überlagerung des sichtbaren Bildes mit CT/MR-Informationen
	    \end{itemize}
        \column{.55\textwidth}
	\pgfimage[width=\textwidth]{Stereodisplay.jpeg}
\end{columns}
\end{frame}

\begin{frame}
\frametitle{Displaytypen (Auszug)}
\begin{itemize}
 \item Optical See-Through $\rightarrow$ Halbdurchlässig
 \item Video See-Through
 \item Head Up Displays (HUDs) (Monokular / Binokular
 \item (Auto)stereoskopische Displays
 \item Virtual Retinal Displays
\end{itemize}
\end{frame}
%
%
\section{ARION\texttrademark - Ein existierendes Visualisationssystem}
\begin{frame}
\frametitle{ARION\texttrademark - Ein existierendes Visualisationssystem}
\begin{itemize}
\item Vorgestellt 2002 von Suthau et al.
\item Benutzt CT- sowie Angiographische Daten
\item Magnetisches Trackingsystem zum Ausgleich der Körperbewegungen und zur Ortung der chirurgischen Instrumente
\item Augmented Reality: Autostereoskopisches Display
\end{itemize}
\end{frame}

\begin{frame}
 \frametitle{Download der Arbeit sowie der Präsentationen auf:}
 \begin{center}
   \textbf{\url{http://tinyurl.com/ulikoehler-seminararbeit}}
 \end{center}
\end{frame}

%
%
\section{Quellen}
\begin{frame}[shrink]
 \frametitle{Bildquellen}
 \begin{itemize}
  \item Hervorhebung von Tumoren:\\ \url{http://radiographics.rsna.org/content/27/3/687/F36.expansion}
  \item Augmented Reality I :\\ \url{http://www.iwb.tum.de/AR_HMI_f\%C3\%BCr_Industrieroboter.print}
  \item Weitere: Selbst erstellt / Public Domain
 \end{itemize}
 
\end{frame}

\end{document}
