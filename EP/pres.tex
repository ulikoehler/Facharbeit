\documentclass[14pt]{beamer}
\usepackage[utf8x]{inputenc}
\usepackage[T1]{fontenc}
\usepackage[ngerman]{babel}
\usepackage{hyperref}
\usepackage{cite}
\usetheme[secheader]{Boadilla}
\usefonttheme{serif}

\title{3D-Tumorvisualisation}
\subtitle{Endpräsentation}
\author{Uli Köhler}
\institute[EMG]{Ernst-Mach-Gymnasium Haar}
\date{8.~Juli 2010}


\AtBeginSection[]{} % for optional outline or other recurrent slide

\begin{document}
\frame{\titlepage}
\begin{frame}
\frametitle{Aufbau der Präsentation}
\tableofcontents
\end{frame}
\section{Behandelte Hauptkonzepte}
%%%%%%
\subsection{Hervorhebung von Tumoren}
\begin{frame}[allowframebreaks]
 \frametitle{Hervorhebung von Tumoren}
    \begin{itemize}
     \item \textbf{Idee:} Mediziner sollen Tumoren schnell erkennen können
     \item Automatisierte Erkennung anhand der HUs\\
	  $\Rightarrow$ Algorithmus: Abbildung der Hounsfield-Skala auf einen Farbverlauf (`Klassifizierung`)
     \item Grenzen des Verfahrens:
      \begin{itemize}
	\item Tumoren mit geringem Kontrast zum umliegenden Gewebe
	\item Menschliche Kenntnis zur Zuordnung von anatomischen Strukturen nötig
      \end{itemize}
    \end{itemize}
    \pgfimage[width=\textwidth]{F36-rendered.png}
\end{frame}
%%%%%%
\subsection{Generierung von Zwischenbildern}
\begin{frame}[allowframebreaks]
 \frametitle{Generierung von Zwischenbildern}
      \begin{block}{Problem}
	      Bildstapel mit geringer Anzahl an Bildern können von der Seite aus nur in ungenügender
	      Auflösung betrachtet werden.
      \end{block}
    \begin{itemize}
     \item Lösungsmöglichkeit: \textit{Interpolation} von Zwischenbildern:
      \begin{itemize}
	\item Automatisch, z.B. Farbverlauf zwischen \textit{Vertices} in OpenGL
	\item Manuell $\rightarrow$ Vorberechnung von virtuellen Bildern
      \end{itemize}
    \end{itemize}
    \pgfimage[width=\textwidth]{interimage1.jpg}
\end{frame}
%%%%%%
\subsection{3D-Navigation}
\begin{frame}
 \frametitle{3D-Navigation}
 \begin{itemize}
  \item Grundoperationen (Transformationen):
  \begin{itemize}
   \item Translation
   \item Rotation
   \item Skalierung
  \end{itemize}
  \item Algorithmen für die Umsetzung Eingabegerät $\rightarrow$ Transformation
  \item Ziel: Intuitive Bedienbarkeit
 \end{itemize}
\end{frame}
%
%
%
\section{Projekt: MediGL}
%
\subsection{Architektur}
\begin{frame}[allowframebreaks]
\frametitle{Architektur}
\pgfimage[width=\textwidth]{ws-1-1.jpg}\\
\pgfimage[width=\textwidth]{ws-1-2.jpg}
\end{frame}
%
\subsection{Fortschritt}
\begin{frame}
\frametitle{Fortschritt}
\begin{itemize}
 \item Implementiert:
  \begin{itemize}
   \item Point Cloud Rendering
   \item 3D-Navigation
   \item DICOM- und Image-Import
  \end{itemize}
  \item Noch fehlend:
  \begin{itemize}
   \item Zwischenbildberechnung
   \item Einschränkung des HU-Intervalls
   \item Weitere Rendering- und Klassifizierungsmethoden
  \end{itemize}
\end{itemize}

\end{frame}
%
%
%
\section{Augmented Reality - Zukunft der Visualisation}
\begin{frame}[allowframebreaks]
\frametitle{Augmented Reality}
\begin{columns}
        \column{.45\textwidth}
	    \begin{itemize}
	    \item `Erweiterte Realität`
	    \item Echtzeiteinblendung von Informationen, z.B. für Chirurgen
	    \item Beispiel: Überlagerung des sichtbaren Bildes mit CT/MR-Informationen
	    \end{itemize}
        \column{.55\textwidth}
	\pgfimage[width=\textwidth]{Stereodisplay.jpeg}
\end{columns}

\pgfimage[width=\textwidth]{ar-full.jpg}
\end{frame}
%
%
%
\section{Literatur}
\begin{frame}[allowframebreaks]
\frametitle{Literatur}
\bibliography{pres}
\bibliographystyle{alpha}
\nocite{*}
\end{frame}
%
%
%
\section{Quellen}
\begin{frame}[shrink]
 \frametitle{Bildquellen}
 \begin{itemize}
  \item Hervorhebung von Tumoren II :\\ \url{http://radiographics.rsna.org/content/27/3/687/F36.expansion}
  \item Augmented Reality I :\\ \url{http://www.iwb.tum.de/AR_HMI_f\%C3\%BCr_Industrieroboter.print}
  \item Augmented Reality II :\\ \url{http://www.stonybrookmedicalcenter.org/pathology/neuropathology/chapter3}
  \item Weitere: Selbst erstellt / Public Domain
 \end{itemize}
 
\end{frame}

\end{document}
