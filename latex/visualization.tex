\documentclass[a4paper]{scrartcl}
\usepackage[utf8x]{inputenc}
\usepackage[T1]{fontenc}
\usepackage[ngerman]{babel}
\usepackage
[colorlinks=true,linkcolor=red,
 anchorcolor=black,citecolor=green,
 pagecolor=red,urlcolor=cyan,backref,]{hyperref}
\usepackage{cite}
\usepackage[german]{varioref}

\title{3D-Tumorvisualisation}
\subtitle{Seminararbeit}
\author{Uli Köhler}
%\institute[EMG]{Ernst-Mach-Gymnasium Haar}
\date{9.~November 2010}

\begin{document}
\maketitle
\tableofcontents \newpage
\section{Einleitung}\label{sec:introduction}
In dieser Arbeit sollen Methoden und Algorithmen dargestellt werden, die zur dreidimensionalen Visualisation von Tumoren in Echtzeit dienen. Hierzu werden zuerst in Kapitel \vref{ssec:requirements} die allgemeinen Anforderungen und in Kapitel \vref{ssec:applications}
Anwendungsmöglichkeiten für solche Systeme dargestellt. Darauf aufbauend werden in Kapitel \vref{ssec:concepts} die Konzepte und Algorithmen dargestellt, die im Rahmen dieser Arbeit entwickelt wurden. Zur anschaulichen Darstellung und als Beweis für die Implementierbarkeit dieser Konzepte in Software wurde begleitend zu dieser Arbeit ein Programm (`VERTEBRA`) geschrieben, anhand dessen in Kapitel \vref{ssec:implementations} Möglichkeiten der Implementation der zuvor dargestellten Konzepte und deren Anwendbarkeit in der modernen Medizin diskutiert werden. Da sich diese Arbeit auf dreidimensionale Visualisationssysteme beschränkt, werden in Kapitel \vref{ssec:3dnav} zusätzlich Verfahren zur Navigation im dreidimensionalen Koordinatensystem vorgestellt, wobei besonderen Wert auf die in Kapitel \ref{ssec:requirements} dargestellten Anforderungen gelegt wird.

Abschließend wird in Kapitel \vref{sec:augmentedreality} die Technik der `Augmented Reality` diskutiert, die in Zukunft eine wichtige Rolle in der medizinischen Visualisationstechnik und speziell im Bereich der Chirurgie spielen könnte.

Aufgrund des geringen Umfangs dieser Arbeit von \pageref{appendixstart} Seiten wurden die Folgenden Themen nicht im Detail behandelt.
\section{Echtzeit-Tumorvisualisationssysteme}\label{sec:vissystems}
\subsection{Anforderungen}\label{ssec:requirements}
\subsection{Anwendungsmöglichkeiten}\label{ssec:applications}
Schon heute werden verschiedene medizinische Visualisationssysteme in den verschiedensten Bereichen der Medizin eingesetzt. 
\section{Konzepte zur Visualisation von Tumoren}\label{ssec:concepts}
\subsection{Implementationsmöglichkeiten am Beispiel des Projektes VERTEBRA}\label{ssec:implementations}
Als Begleitprojekt zu dieser Arbeit wurde das Programm VERTEBRA
(Volumetric Examiner for Radiological and Tomographical Experimental Basic Realtime Analysis) geschrieben, dessen Hauptzweck darin besteht, die Machbarkeit in dieser Arbeit vorgestellten Konzepte zu demonstrieren und im Rahmen der erreichbaren Genauigkeit vergleichbare Daten über die Echtzeittauglichkeit der Algorithmen und Verfahren zu liefern \\
VERTEBRA ist ein Proof-of-Concept-Projekt, zielt also nicht darauf ab, ein im medizinischen Alltag einsetzbares Produkt zu sein oder die für eine Zulassung nach dem Medizinproduktegesetz (bezogen auf deutsches Recht) notwendigen Vorraussetzungen zu schaffen.
\subsection{Algorithmen zur Navigation im 3D-Koordinatensystem}\label{ssec:3dnav}
\subsection{Grenzen der vorgestellten Konzepte}\label{ssec:limits}
\paragraph{Abhängigkeit von den Eingabedaten}
Die in dieser Arbeit vorgestellten Verfahren benutzen mathematische Verfahren, um eine Eingabedatenmenge variabler Größe in eine Menge an Eingabedaten zu übersetzen, die von spezialisierter Grafikhardware mathematisch in ein zweidimensionales Ausgangsbild umgerechnet wird, um auf einem visuellen Ausgabegerät, zum Beispiel einem Monitor angezeigt zu werden. Aufgrund dieser mathematischen Umrechnung basieren die Ausgabedatenmengen dieser Algorithmen ausschließlich auf den Eingabedatenmengen, die üblicherweise Datensätze aus tomografischen Scannern repräsentieren und den Konfigurationsparametern der Algorithmen.
\subsection{Vergleich von Methoden zur Berechnung von Visualisationsdaten}
\subsubsection{Vergleich von Software- und Hardwarealgorithmen}\label{ssec:swhwcomparison}
\subsubsection{Berechnung auf dem Hauptprozessor}\label{sssec:cpucalculation}
Die naheliegendste Möglichkeit, Operationen auf großen Mengen volumetrischen Daten durchzuführen, ist, die Berechnungen vom Hauptprozessor des Computers durchführen zu lassen. 
\subsubsection{Berechnung auf GPUs}\label{sssec:gpucalculation}
Da die Rechenleistung der Hauptprozessoren moderner Computer für viele der heutigen 3D-Anwendungen nicht mehr ausreicht, enthält ein Grossteil der heutigen Computer Grafikhardware, die in der Lage ist, 3D-Anwendungen zu beschleunigen. Hierbei berechnet der Hauptprozessor die darzustellenden Daten und gibt sie an die Grafikhardware weiter, die diese Daten mithilfe der GPU\footnote{GPU - Graphics Processing Unit - Grafikprozessor} in ein zweidimensionales Bild umwandelt (rendert), das beispielsweise auf einem Monitor angezeigt werden kann.

Um der GPU mitzuteilen, welche Objekte wo im dreidimensionalen Koordinatensystem auf welche Art gerendert werden sollen, müssen Programme ein Grafik-API\footnote{API - Application Programming Interface - Programmierschnittstelle} wie OpenGL (\cite{OpenGLWebsite}) oder DirectX (\cite{DirectXWebsite})

GPUs neueren Datums können ausserdem so genannte Shader ausführen - kleine Programme, die für bestimmte Untereinheiten der zu visualisierenden Objekte ausgeführt werden. Die Shader-Prozessoren (Untereinheiten der GPU
\subsubsection{Berechnung auf FPGAs}
Abgesehen von den bereits diskutierten Möglichkeiten ist mit so genannten FPGAs\footnote{FPGA - Field Programmable Gate Array} eine weitere Form der Hardwarebeschleunigung für die Medizinische Visualisationstechnik vorhanden. FPGAs sind ICs\footnote{IC - Integrated Circuit - Integrierter Schaltkreis}, bei denen nach der Herstellung eine anwendungsspezifische Programmierung und Konfiguration möglich ist (\cite{Kibritev2009}, Kapitel 1.7.1, Seite 16). Daher sind die Produktionskosten für ein auf FPGAs basierendes Visualisationssystem wesentlich geringer als bei dedizierter Hardware, die für eine bestimmte Aufgabe gefertigt wurde und nicht nachträglich programmierbar ist. Wie bereits in Kapitel \vref{ssec:swhwcomparison} diskutiert stellt Bruckner in \cite{Bruckner2008} Nachteile von hardwarebasiertem Rendering gegenüber softwarebasierten Darstellungen in Bezug auf große volumetrische Datensätze dar. FPGAs erlauben, wie Leeser et al. in \cite{Leeser2005} mithilfe einer Implementation des Parallel-Beam Backprojection-Algorithmus zeigen, eine performante Implementation dieser Softwarealgorithmen. In \cite{Thomas2009} findet sich ein Vergleich die Performanz von CPU-, GPU- und FPGA-basierten Systemen anhand von Zufallszahlenalgorithmen - wie aus Tabelle 6 ersichlich liefert das im Experiment benutzt FPGA-basierte System bei zwei der drei dargestellen Verteilungen, die jeweils unterschiedliche Algorithmen benötigen bessere Resultate (also eine größere Zahl generierter Zufallszahlen pro Zeiteinheit) als die anderen getesteten Plattformen.
\subsection{Augmented Reality - Zukunft von Visualisationssystemen}\label{sec:augmentedreality}
In den letzten Jahren wird neben den herkömmlichen Geräten zu Darstellung der Informationen medizinischer Informationssysteme (z.B. Bildschirme) intensiv an der so genannten `Augmented Reality` geforscht. Dieser Begriff, der sich mit 'Erweiterte Realität' (Quelle: \cite{Toe2010}, Seite 1) übersetzen lässt, beschreibt laut \cite{Suthau2002DE} (Seite 1; Englische Publikation: \cite{Suthau2002}) das Konzept, reale Bilder mit zusätzlichen Informationen zu ergänzen. 
\appendix \label{appendixstart}
\section{Anleitung zur Übersetzung von VERTEBRA}
Um den Quellcode von VERTEBRA in Maschinencode zu übersetzen, werden die folgenden Hilfsprogramme bzw. Bibliotheken benötigt:
\begin{itemize}
 \item C++-Compiler kompatibel zu ISO/IEC C++ 2003
 \item Qt Software Development Toolkit (SDK): minimal Version 4.7.0 = Qt SDK 2009.05.1
 \item DCMTK 3.5.4
\end{itemize}

Das Qt SDK kann von \cite{QtWebsite} bezogen werden; die DCMTK-Bibliothek, die zum einlesen von DICOM-Daten benutzt wird, wird vom OFFIS e.v. (Oldenburger Forschungs- und Entwicklungsinstitut für Informatik, \cite{OFFISWebsite}) auf \cite{DCMTKWebsite} zu Verfügung gestellt.

Das Programm wurde mit der folgenden Konfiguration erfolgreich getestet (siehe auch Kapitel \vref{apdx:testplatform}):
\begin{itemize}
 \item C++-Compiler: GNU Compiler Collection \textit{gcc (Ubuntu/Linaro 4.4.4-14ubuntu5) 4.4.5}
 \item Qt SDK: 4.7.00 = Qt SDK 2009.05.1
\end{itemize}

\section{Testplattform}\label{apdx:testplatform}
Die in dieser Arbeit vorgestellten Geschwindigkeitsmessungen wurden auf der folgenden Plattform durchgeführt:
\begin{itemize}
  \item Dell Inspiron 530
  \item CPU: Intel\textsuperscript{\textregistered} Core\texttrademark 2 Duo E8300; 2.83GHz Taktfrequenz
  \item Betriebssystem: KUbuntu 10.10 x86\_64; Kernel 2.6.35-22-generic x86\_64
  \item Grafikhardware: NVidia\textsuperscript{\textregistered} GeForce 9400 GT, PCI Express x16\\
	Treiber 260.19.06 (Konfiguration: High Performance)
\end{itemize}
\section{Quellcode von VERTEBRA}
\section{Doxygen-Dokumentation von VERTEBRA}
\section{Selbstständigkeitserklärung}
Hiermit erkläre ich, dass ich die vorliegende Arbeit in allen Teilen selbstständig verfasst habe und keine anderen als die angegebenen Quellen und Hilfsmittel (einschließlich elektronischer Medien und Onlinequellen) benutzt habe.
\newpage
\section{Literatur- und Quellenverzeichnis}
%\nocite{*}
\renewcommand\refname{Literatur- und Quellenverzeichnis}
\bibliographystyle{plaindin}
\bibliography{visualization}
\end{document}
