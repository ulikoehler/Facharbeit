\documentclass[a4paper]{scrartcl}
\usepackage[utf8x]{inputenc}
\usepackage[T1]{fontenc}
\usepackage[ngerman]{babel}
\usepackage{hyperref}
\usepackage{cite}

\title{3D-Tumorvisualisation}
\subtitle{Seminararbeit}
\author{Uli Köhler}
%\institute[EMG]{Ernst-Mach-Gymnasium Haar}
\date{9.~November 2010}

\begin{document}
\tableofcontents
\section{Einleitung}
\subsection{Themenstellung}
\subsection{Ziele der Arbeit}
\section{Echtzeit-Tumorvisualisationssystem}
\subsection{Anforderungen}
\subsection{Anwendungsmöglichkeiten}
\section{Algorithmen und Methoden am Beispiel des Projektes VERTEBRA}
\subsection{Konzepte zur Hervorhebung von Tumoren}
\subsection{Algorithmen zur Navigation im 3D-Koordinatensystem}
\subsection{Vergleich von Methoden zur Berechnung von Visualisationsdaten über große Datenmengen}
\subsubsection{CPU-Vorabberechnung}
\subsubsection{Berechnung mithilfe von GLSL-Shadern}
\subsubsection{}
\subsection{Augmented Reality - Zukunft von Visualisationssystemen}
In letzter Zeit zeigt sich vor allem durch die große Zahl der Publikationen über dieses Thema, dass sich die Zunkunft von medizinischen
Visualisationssystemen in der Augmented Reality befindet. Dieser Begriff, der für virtuelle Realität \cite{Botden2009}
\appendix
\section{Quellenverzeichnis}
\section{Quellcode von VERTEBRA}
\section{Anleitung zur Übersetzung von VERTEBRA}
\section{Doxygen-Dokumentation von VERTEBRA}
\section{Selbstständigkeitserklärung}
\bibliographystyle{apalike}
\bibliography{visualization}
\end{document}
